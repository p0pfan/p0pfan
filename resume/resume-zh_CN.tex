% !TEX TS-program = xelatex
% !TEX encoding = UTF-8 Unicode
% !Mode:: "TeX:UTF-8"

\documentclass{resume}
\usepackage{zh_CN-Adobefonts_external} % Simplified Chinese Support using external fonts (./fonts/zh_CN-Adobe/)
%\usepackage{zh_CN-Adobefonts_internal} % Simplified Chinese Support using system fonts
\usepackage{linespacing_fix} % disable extra space before next section
\usepackage{cite}

\begin{document}
\pagenumbering{gobble} % suppress displaying page number

\name{陈诚}

% {E-mail}{mobilephone}{homepage}
% be careful of _ in emaill address
\contactInfo{ch.ch@outlook.com}{(+86) 131-277-37652}{12/27/1991}
% {E-mail}{mobilephone}
% keep the last empty braces!
%\contactInfo{xxx@yuanbin.me}{(+86) 131-221-87xxx}{}

\section{\faGraduationCap\  教育背景}
\datedsubsection{\textbf{华东理工大学}, 上海}{2014 -- 至今}
\textit{在读硕士研究生}\ 控制科学与工程, 预计 2017 年 6 月毕业

\datedsubsection{\textbf{山东理工大学}, 淄博, 山东}{2010 -- 2014}
\textit{学士}\ 自动化
\begin{itemize}
\item \role{ 平均学分绩点: 88.725, 专业排名: 3/169}{}

\end{itemize}

\section{\faUsers\ 实习经历}
\datedsubsection{\textbf{美光半导体(Micron)} 上海}{2015年8月 -- 2014年4月}
\role{实习}{}
1.设备管理工具的开发
\begin{itemize}
  \item 开发语言:\emph{Python}
  \item 通过\emph{Jenkins}实现对设备的当前的状态进行管理,记录。
  \\
\end{itemize}
2.成员任务的Timeline的开发
\begin{itemize}
  \item 开发语言:\emph{Python}
  \item 利用\emph{Jira}的API和\emph{Jenkins}的插件实现在HTML中显示;
  \item 实现对小组成员的当日任务在时间线中显示;
  \item 并实现Jira Ticket的创建。
  \\
\end{itemize}

3.组织了一次Team Building活动。
 


% Reference Test
%\datedsubsection{\textbf{Paper Title\cite{zaharia2012resilient}}}{May. 2015}
%An xxx optimized for xxx\cite{verma2015large}
%\begin{itemize}
%  \item main contribution
%\end{itemize}

\section{\faCogs\ 专业技能}
% increase linespacing [parsep=0.5ex]
\begin{itemize}[parsep=1ex]
  \item 熟悉\emph{Python}、\emph{Java}、\emph{C}语言
  \item 了解汇编语言、\emph{Git}、\emph{SQL}、\emph{Jenkins}
  \item 会使用简单的\emph{Git}指令管理代码
  \item 会使用\emph{Jenkins}建立简单的Job
  \item 会使用\LaTeX 进行论文排版 
  \item 能熟练阅读英文文献资料,具有一定的英文写作和日常口语表达能力,CET-4:508,CET-6:484。

\end{itemize}

\section{\faHeartO\ 获奖情况}
\begin{itemize}[parsep=0.5ex]
\datedline{\item 本科期间连续4个学年获得校级优秀学生奖学金}{}
\datedline{\item 获校级三好学生}{2013-2014学年}
\datedline{\item 获得研究生学业一等奖学金}{2014-2015学年}

\end{itemize}

\section{\faInfo\ 自我评价}
% increase linespacing [parsep=0.5ex]
\begin{itemize}[parsep=0.5ex]
  \item 有较强的好奇心,对互联网技术非常感兴趣;
  \item 与人友好相处,为人坦诚,踏实肯干,努力刻苦,责任心强,具有团队合作精神;
  \item 善于接受新事物,具有较强独立思考,自学能力。
\end{itemize}

%% Reference
%\newpage
%\bibliographystyle{IEEETran}
%\bibliography{mycite}
\end{document}
